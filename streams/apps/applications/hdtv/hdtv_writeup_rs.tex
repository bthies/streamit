\subsection{Overview}
Reed-Solomon codes are a type of forward error correcting code that are used in everything from CDs to
satellite communications. The theoretical foundations for these codes were published in the 
in 1960\cite{reed-solomon}. The authors, Irving S. Reed and Gustave Solomon, were researchers
at MIT's Lincoln Labs, and they were interested in the newly created field of information theory
and information coding. For an
interesting history of the reed-solomon codes, an excellent overview and brief history can be
found in \cite{reed-solomon-overview}.

In coding parlance, the sender is the person who has data that they want to protect by means
of the code. The sender encodes their data with the code, which results in a codeword.
The codeword is then sent over a (possibly) noisy channel which has the potential to 
add errors to the code word. The reciever is the person who recieves (imagine that)
the codeword from the channel, and can then use the code and the recieved codeword to
try and figure out the data that the sender originally intended to send.

Note that the sender/channel/receiver model is applicable to a wide set of scenarios, not
just communication systems where you really do have a sender, a channel (eg radio waves), and
a recieved. For instance, a CD also falls under this model. The sender is the record publisher
who wants to send digital audio. They encode their audio data onto a CD, which then acts as 
the channel between the record company and you. When you buy a CD and give it a scratch or two,
the distortions to the CD can be thought of as a channel adding noise. It is then up to your CD 
player, as the receiver, to take the scratched CD and produce the original audio that the
record company placed there.

If a code is ``Forward Error Correcting,'' or FEC, it means that the person who is trying to figure out
what the encoded data represents is not allowed to communicate with the sender. For example, when you
have a scratch on your CD, your CD player can't ask the CD's publisher what byte number 2243532 was supposed
to be if there were too many errors. However, if your computer wants to recieve a web page, and one of the packets
from the webserver gets lost in transit, it \textbf{can} request that the webserver resend that particular
packet.

Because an FEC has to make do with one way communication, redundancy is added to the original data to form the
code word. The hope is that even if some of the codeword is changed in transit over the channel the
added redundancy will let the receiver figure out what was supposed to be sent. Reed-Solmon codes are
especially cool because 

\subsection{RS Encoding}

\subsection{RS Decoding}



\subsection{StreamIT Implementation}
\begin{figure}
\center
\epsfxsize=2.5in
\epsfbox{streamgraphs/SGRSEncoder.eps}
\epsfxsize=2.5in
\epsfbox{streamgraphs/SGRSDecoder.eps}
\caption{StreamGraphs of the ReedSolomonEncoder(left) and the ReedSolomonDecoder(right).}
\label{fig:sg-reed-solomon}
\end{figure}

Reed-Solomon encoders and decoders are far from trivial to implement. 
They need implementations of Galois (finite field) arithemetic, generator polynomials, 
and other exotic mathematics that are not in the repitoire of most programmers.

To implement the Reed-Solomon coders for HDTV, we pressed an open
source implementation of Reed-Solomon encoding, RSCODE\cite{rscode}
 into service. RSCODE was written in C, but converting the C implementation
to the Java like syntax of StreamIT was fairly straightforward. Some global variables 
needed to be converted into fields, and we needed to change the data types of some 
of the internal arrays so that they would compile successfully. It seems as though
Java and C have differing symantics on what values a \texttt{char} character
can take on.

Figure~\ref{sg-reed-solomon} shows the StreamIT stream graph of the Reed-Solomon
decoder. Since the code was originally a C library, the easiest (and the only obvious)
way to implement the encoder and decoder in streamit was to have a monolothic filter
which did everything in one go. While this doesn't expose much parallelism to the
StreamIT compiler, it is not clear that there really is all that much parallelism in the
application to expose. 